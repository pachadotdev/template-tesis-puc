%\documentclass[12pt,reqno,oneside]{puctesis}         % For dvips
\documentclass[12pt,reqno,oneside,pdftex]{puctesis} % For pdflatex
%\documentclass[10pt,reqno,twoside]{puctesis}
%\draft
%\doublespacing
%\usepackage{verbatim}
%\usepackage{setspace}
\usepackage{graphicx}
\usepackage{amsmath}
\usepackage{amsfonts}
\usepackage{amssymb}
\usepackage[spanish]{algorithm2e}
\usepackage{fancybox}
\usepackage{float}
\usepackage{times}

\usepackage{color}
\usepackage{fancyvrb}
\newcommand{\VerbBar}{|}
\newcommand{\VERB}{\Verb[commandchars=\\\{\}]}
\DefineVerbatimEnvironment{Highlighting}{Verbatim}{commandchars=\\\{\}}
\usepackage{framed}
\definecolor{shadecolor}{RGB}{248,248,248}
\newenvironment{Shaded}{\begin{snugshade}}{\end{snugshade}}
\newcommand{\AlertTok}[1]{\textcolor[rgb]{0.94,0.16,0.16}{#1}}
\newcommand{\AnnotationTok}[1]{\textcolor[rgb]{0.56,0.35,0.01}{\textbf{\textit{#1}}}}
\newcommand{\AttributeTok}[1]{\textcolor[rgb]{0.77,0.63,0.00}{#1}}
\newcommand{\BaseNTok}[1]{\textcolor[rgb]{0.00,0.00,0.81}{#1}}
\newcommand{\BuiltInTok}[1]{#1}
\newcommand{\CharTok}[1]{\textcolor[rgb]{0.31,0.60,0.02}{#1}}
\newcommand{\CommentTok}[1]{\textcolor[rgb]{0.56,0.35,0.01}{\textit{#1}}}
\newcommand{\CommentVarTok}[1]{\textcolor[rgb]{0.56,0.35,0.01}{\textbf{\textit{#1}}}}
\newcommand{\ConstantTok}[1]{\textcolor[rgb]{0.00,0.00,0.00}{#1}}
\newcommand{\ControlFlowTok}[1]{\textcolor[rgb]{0.13,0.29,0.53}{\textbf{#1}}}
\newcommand{\DataTypeTok}[1]{\textcolor[rgb]{0.13,0.29,0.53}{#1}}
\newcommand{\DecValTok}[1]{\textcolor[rgb]{0.00,0.00,0.81}{#1}}
\newcommand{\DocumentationTok}[1]{\textcolor[rgb]{0.56,0.35,0.01}{\textbf{\textit{#1}}}}
\newcommand{\ErrorTok}[1]{\textcolor[rgb]{0.64,0.00,0.00}{\textbf{#1}}}
\newcommand{\ExtensionTok}[1]{#1}
\newcommand{\FloatTok}[1]{\textcolor[rgb]{0.00,0.00,0.81}{#1}}
\newcommand{\FunctionTok}[1]{\textcolor[rgb]{0.00,0.00,0.00}{#1}}
\newcommand{\ImportTok}[1]{#1}
\newcommand{\InformationTok}[1]{\textcolor[rgb]{0.56,0.35,0.01}{\textbf{\textit{#1}}}}
\newcommand{\KeywordTok}[1]{\textcolor[rgb]{0.13,0.29,0.53}{\textbf{#1}}}
\newcommand{\NormalTok}[1]{#1}
\newcommand{\OperatorTok}[1]{\textcolor[rgb]{0.81,0.36,0.00}{\textbf{#1}}}
\newcommand{\OtherTok}[1]{\textcolor[rgb]{0.56,0.35,0.01}{#1}}
\newcommand{\PreprocessorTok}[1]{\textcolor[rgb]{0.56,0.35,0.01}{\textit{#1}}}
\newcommand{\RegionMarkerTok}[1]{#1}
\newcommand{\SpecialCharTok}[1]{\textcolor[rgb]{0.00,0.00,0.00}{#1}}
\newcommand{\SpecialStringTok}[1]{\textcolor[rgb]{0.31,0.60,0.02}{#1}}
\newcommand{\StringTok}[1]{\textcolor[rgb]{0.31,0.60,0.02}{#1}}
\newcommand{\VariableTok}[1]{\textcolor[rgb]{0.00,0.00,0.00}{#1}}
\newcommand{\VerbatimStringTok}[1]{\textcolor[rgb]{0.31,0.60,0.02}{#1}}
\newcommand{\WarningTok}[1]{\textcolor[rgb]{0.56,0.35,0.01}{\textbf{\textit{#1}}}}

% --- START: Babel ---
% Esta seccion es necesaria para quienes desean usar Babel y acentos
% en castellano en vez de acentos LaTeX tradicionales, e.j. \'a, \'e, etc.
% Como Babel tiene un manejo distinto al normal para definir los nombres
% de las partes del documento no basta con usar \renewcommand{\...name}
% sino que es necesario usar tambi\'en \addto\captionspanish.
% MTT 2011.11.30
\usepackage[spanish]{babel}    % Estos paquetes alteran las definiciones
\usepackage[utf8]{inputenc}
%\usepackage[latin1]{inputenc}  % estandar de los titulos y requieren 
                               % redefiniciones adicionales
\addto\captionsspanish{%
  \renewcommand{\contentsname}{INDICE GENERAL}%
}
\addto\captionsspanish{
 \renewcommand{\listfigurename}{INDICE DE FIGURAS}%
}
\addto\captionsspanish{
 \renewcommand{\listtablename}{INDICE DE TABLAS}%
}
\addto\captionsspanish{
 \renewcommand{\chaptername}{}%CAPITULO}
}
\addto\captionsspanish{
 \renewcommand{\figurename}{Figura}%
}
\addto\captionsspanish{
 \renewcommand{\tablename}{Tabla}%
}
\addto\captionsspanish{
 \renewcommand{\refname}{BIBLIOGRAFIA}%
}
\addto\captionsspanish{
 \renewcommand{\bibname}{}%
}
\addto\captionsspanish{
 \renewcommand{\BOthers}[1]{et al.\hbox{}}%
}
% --- END: Babel ---

           %%%%%%%%%%%%%%%%%%%%%%%%%%%%%%%%%%%%%%%%%%%%%%%%%%%%
           %   Preambulo                                      %
           %------------------------------------------------- %
           %        \newcommand\...{...}                      %
           %        \newtheorem{}{}[]                         %
           %        \numberwithin{}{}                         %
           %%%%%%%%%%%%%%%%%%%%%%%%%%%%%%%%%%%%%%%%%%%%%%%%%%%%


%--------- NUEVOS ENTORNOS ---------
\newtheorem{definicion}{\bf Definici\'on}[chapter]
\newtheorem{propiedad}{Propiedad}[chapter]
\newtheorem{afirmacion}{Afirmaci\'on}[chapter]
\newtheorem{lema}{\bf Lema}[chapter]
\newtheorem{proposicion}{Proposici\'on}[chapter]
\newtheorem{teorema}{\noindent \bf Teorema}[chapter]
\newtheorem{corolario}{\bf Corolario}[chapter]
\newtheorem{pf}{Demostraci\'on}[chapter]
\newtheorem{ejemplo}{\bf Ejemplo}[chapter]
\newtheorem{comentario}{Comentario}[chapter]

%--------- COLOQUE ENTORNOS/DEFINICIONES ADICIONALES AQUI ---------

\newcommand\opgrad{\operatorname{grad}}        
% ...


%----------------------------------------------------------------------%
\begin{document}

           %%%%%%%%%%%%%%%%%%%%%%%%%%%%%%%%%%%%%%%%%%%%%%%%%%%%
           %                                                  %
           %  INICIALIZACIONES : PORTADA                      %
           %                                                  %
           %%%%%%%%%%%%%%%%%%%%%%%%%%%%%%%%%%%%%%%%%%%%%%%%%%%%
%\draft                        %a?ade nota al pie con fecha del borrador
\mdate{April 17, 2007}         %fecha de modificacion del manuscrito
\version{1}                    %numero de version del manuscrito


$if(titulo)$
\title{$titulo$}
$endif$
$if(autor)$
\author{$autor$}
$endif$

$if(escuelaodepartamento)$
\facultyto    {$escuelaodepartamento$}
$endif$
$if(facultad)$
\faculty      {$facultad$}
\address{$facultad$\\
         Pontificia Universidad Cat\'olica de Chile\\ 
         Vicu\~na Mackenna 4860\\
         Santiago, Chile\\
         {\it Tel.\/} : 56 (2) 354-2000}
$endif$
$if(tituloogrado)$
\degree       {$tituloogrado$} 
$endif$
$if(profesorguia)$
\advisor      {$profesorguia$}
$endif$
$if(miembrocomite1)$
\committeememberA {$miembrocomite1$}
$endif$
$if(miembrocomite2)$
\committeememberB {$miembrocomite2$}
$endif$
$if(miembroinvitadocomite1)$
\guestmemberA {$miembroinvitadocomite1$}
$endif$
$if(miembroinvitadocomite2)$
\guestmemberB {$miembroinvitadocomite2$}
$endif$
$if(representantedipei)$
\ogrsmember   {$representantedipei$}
$endif$
$if(materia)$
\subject      {$materia$}
$endif$
$if(fecha)$
\date         {$fecha$}
$endif$
$if(nombrecompleto)$
\copyrightname{$nombrecompleto$}
$endif$
$if(anioromano)$
\copyrightyear{$anioromano$}
$endif$
$if(dedicatoria)$
\dedication   {$dedicatoria$}
$endif$

           %%%%%%%%%%%%%%%%%%%%%%%%%%%%%%%%%%%%%%%%%%%%%%%%%%%%
           %   PRELIMINARIDADES                               %
           %--------------------------------------------------%
           %      pags. i & ii: segunda pagina                %
           %      pags. iii: dedicatoria                      %
           %%%%%%%%%%%%%%%%%%%%%%%%%%%%%%%%%%%%%%%%%%%%%%%%%%%%

\PageNumbersFootCentered       % numeros de pagina centrados en el pie															
\pagenumbering{roman}
\maketitle


           %%%%%%%%%%%%%%%%%%%%%%%%%%%%%%%%%%%%%%%%%%%%%%%%%%%%
           %   PAGINAS EXTRA                                  %
           %--------------------------------------------------%
           %      pags. --: not used                          %
           %%%%%%%%%%%%%%%%%%%%%%%%%%%%%%%%%%%%%%%%%%%%%%%%%%%%

%\newpage
%\thispagestyle{empty}

%----------------------------------------------------------------------%

           %%%%%%%%%%%%%%%%%%%%%%%%%%%%%%%%%%%%%%%%%%%%%%%%%%%%
           %      pags. iv: AGRADECIMIENTOS                   %
           %%%%%%%%%%%%%%%%%%%%%%%%%%%%%%%%%%%%%%%%%%%%%%%%%%%%

\chapter*{AGRADECIMIENTOS}
Esta plantilla de R Markdown se base en la plantilla Latex hecha por
Miguel Torres Torriti. Los agradecimientos se editan directamente en formato-puc.tex.
\par
%\bigskip

%................................

\cleardoublepage % En la impresion en doble cara, este comando hace que
                 % la siguiente pagina sea una pagina derecha
                 % (es decir, un pagina con numero impar con respecto
                 % a la cuenta absoluta), produciendo una pagina en blanco
                 % si es necesario.  Agregado por MTT 20.AUG.2002

%----------------------------------------------------------------------%

           %%%%%%%%%%%%%%%%%%%%%%%%%%%%%%%%%%%%%%%%%%%%%%%%%%%%
           %      pags. v & up ---                            %
           %            Indice General                        %
           %            Lista de Figuras                      %
           %            Lista de Tablas                       %
           %%%%%%%%%%%%%%%%%%%%%%%%%%%%%%%%%%%%%%%%%%%%%%%%%%%%

\tableofcontents
\listoffigures          
\listoftables           
\cleardoublepage % En la impresion en doble cara, este comando hace que
                 % la siguiente pagina sea una pagina derecha
                 % (es decir, un pagina con numero impar con respecto
                 % a la cuenta absoluta), produciendo una pagina en blanco
                 % si es necesario.  Agregado por MTT 20.AUG.2002

%----------------------------------------------------------------------%

           %%%%%%%%%%%%%%%%%%%%%%%%%%%%%%%%%%%%%%%%%%%%%%%%%%%%
           %      pags. x & xi: RESUMEN - ABSTRACT
           %%%%%%%%%%%%%%%%%%%%%%%%%%%%%%%%%%%%%%%%%%%%%%%%%%%%


\chapter*{RESUMEN}

El resumen del trabajo se edita directamente en formato-puc.tex.

\cleardoublepage % In double-sided printing style makes the next page 
                 % a right-hand page, (i.e. a truly odd-numbered page 
                 % with respect to absolut counting), producing a blank
                 % page if necessary. Added by MTT 20.AUG.2002 

%======================================================================%

           %%%%%%%%%%%%%%%%%%%%%%%%%%%%%%%%%%%%%%%%%%%%%%%%%%%%
           %   TEXTO DE LA TESIS
           %%%%%%%%%%%%%%%%%%%%%%%%%%%%%%%%%%%%%%%%%%%%%%%%%%%%

\NoChapterPageNumber           % elimina encabezado - pie de pagina de la
                               % primera pagina de cada capitulo
\pagenumbering{arabic}

           %%%%%%%%%%%%%%%%%%%%%%%%%%%%%%%%%%%%%%%%%%%%%%%%%%%%
           %   TEXTO DE LA TESIS
           %%%%%%%%%%%%%%%%%%%%%%%%%%%%%%%%%%%%%%%%%%%%%%%%%%%%

$body$

%----------------------------------------------------------------------%


           %%%%%%%%%%%%%%%%%%%%%%%%%%%%%%%%%%%%%%%%%%%%%%%%%%%%
           %   REFERENCIAS 
           %%%%%%%%%%%%%%%%%%%%%%%%%%%%%%%%%%%%%%%%%%%%%%%%%%%%

$if(formatearbibliografia)$
\cleardoublepage
\nocite{*} % To make all the uncited references to appear in the bibliography.
\bibliographystyle{apacite} 
\bibliography{$bibliography$}
$endif$

%----------------------------------------------------------------------%


%----------------------------------------------------------------------%

           %%%%%%%%%%%%%%%%%%%%%%%%%%%%%%%%%%%%%%%%%%%%%%%%%%%%
           %   INDEX 
           %%%%%%%%%%%%%%%%%%%%%%%%%%%%%%%%%%%%%%%%%%%%%%%%%%%%

%% Uncomment the following lines to include an index.

%% INSERT INDEX PAGE # IN TOC
%%%\addtocounter{chapter}{1}
%%%\addcontentsline{toc}{chapter}{\protect\numberline{\thechapter}{Index}}
%%\addcontentsline{toc}{chapter}{\protect\numberline{}{Index}}
%% NOTE: Insert "\label{IDX}" in '.ind' file after compiling the index
%% with makeindex.
%%\index{ @\label{IDX}}
%% The above NOTE is not really needed as can be achieved by the trick below.
%\addtocounter{page}{1}
%\label{IDX}
%\addtocounter{page}{-1}
%\printindex

%----------------------------------------------------------------------%


\end{document}
%======================================================================%
%%%%%%%%%%%%%%%%%%%%%%%%%%%%%%%%%%%%%%%%%%%%%%%%%%%%%%%%%%%%%%%%%%%%%%%%
%% Local Variables:
%% TeX-command-default: "LaTeX"
%% End:
